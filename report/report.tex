\documentclass{llncs}

\usepackage{microtype}
\usepackage{xspace}

\newcommand{\KeY}{Ke\kern-1ptY\xspace}

\title{The \KeY Approach on Hagrid
  \\{\small VerifyThis Long-Term Challenge 2020 }}

\author{ Stijn de Gouw \and Mattias Ulbrich \and Alexander Weigl }
\institute{Open University \and Karlsruhe Institute of Technology}
\begin{document}
\maketitle

\section{Introduction}

1. Motivation

2. VerifyThis Challenge LTC

3. Verification target 

For 2020, the challenge system is the new, implementation of the PGP server
infrastructure, called HAGRID [2]. The old implementation did not conform to
GDPR and was known to be vulnerable against DoS attacks.
    

   % * ... the used verification approach and tools
   %  * ... how the challenge was adapted to make verification possible
   %    (abstractions, reimplementation, different behaviour)
   %  * ... what has been achieved (modelled and verified properties)
   %  * ... successes and challenges encountered in the course of the case
   %    study.

\section{Introduction of the \KeY tool}

Deductive Verification of Java Programs; Supports Java 1.4 and JavaCard;
ultimate precise tool; requires a lot of specification and effort but guarantees
correctness;


\section{Re-implementation of Hagrid in Java}

how the challenge was adapted to make verification possible
(abstractions, reimplementation, different behaviour)

1. We need to reimplement in Java.

2. Decided for a simple implementation, one class only.

3. We verify two different version,

a. a simplified version without the use of any complex methods

b. a complex, map-based version.

4. 

\section{Proofed Properties}

what has been achieved (modelled and verified properties)


\section{Conclusion }

.. successes and challenges encountered in the course of the case study.

\end{document}
