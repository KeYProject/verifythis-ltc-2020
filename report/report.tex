\documentclass{llncs}

\usepackage{microtype}
\usepackage{xspace}

\newcommand{\KeY}{Ke\kern-1ptY\xspace}

\title{The \KeY Approach on Hagrid
  \\{\small VerifyThis Long-Term Challenge 2020 }}

\author{ Stijn de Gouw \and Mattias Ulbrich \and Alexander Weigl }
\institute{Open University \and Karlsruhe Institute of Technology}
\begin{document}
\maketitle

\section{Introduction}

1. Motivation

2. VerifyThis Challenge LTC

3. Verification target 

For 2020, the challenge system is the new, implementation of the PGP server
infrastructure, called HAGRID [2]. The old implementation did not conform to
GDPR and was known to be vulnerable against DoS attacks.
    

   % * ... the used verification approach and tools
   %  * ... how the challenge was adapted to make verification possible
   %    (abstractions, reimplementation, different behaviour)
   %  * ... what has been achieved (modelled and verified properties)
   %  * ... successes and challenges encountered in the course of the case
   %    study.

\section{Introduction of the \KeY tool}

Deductive Verification of Java Programs; Supports Java 1.4 and JavaCard;
ultimate precise tool; requires a lot of specification and effort but guarantees
correctness;


\section{Re-implementation of Hagrid in Java}

how the challenge was adapted to make verification possible
(abstractions, reimplementation, different behaviour)

1. We need to reimplement in Java.

2. Decided for a simple implementation, one class only.

3. We verify two different version,

a. a simplified version without the use of any complex methods

b. a complex, map-based version.

4. 

\paragraph{A simply email-key map}

In simplified/, you find a simple implementation for a email-key map --
verifiable in KeY. The implementation is based upon two arrays, for the key and
the value. The number of pairs are limited as the arrays are never resized. It
also does not support the verifying part of the Verifying Key Server

\paragraph{Map version}

The next version builds upon the generalization of the previous map structure.
It is currently work in progress. This is a two-step process. The first step is
a provable version using maps of int to int. This avoids working with the heap.

KIMap (Key Integer Map) is an interface representing a map of Int -> Int. This
functionality is bound to the behaviour of the map data type in KeY by JML
specification. KIMapImpl is a simple implementation based upon two int arrays,
one for the key, the other the values. KeyServerInt is a version of the backend
of the verifying key server using integers as e-mail addresses and keys. The
second step is to use Strings. This results into KSMap and KSMapImpl and also
the KeyServerString.

\section{Proofed Properties}

what has been achieved (modelled and verified properties)


\section{Conclusion }

.. successes and challenges encountered in the course of the case study.

\end{document}
